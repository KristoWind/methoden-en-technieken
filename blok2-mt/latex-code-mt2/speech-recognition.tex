{\large
\textbf{{\LARGE Bonus: spraakherkenning}}
\section{Spraak}
\begin{itemize}
    \item stemhebbende letters: [z], [v], [b], [d]
    \item stemloze letters: [s], [f], [p], [t]
\end{itemize}

\begin{figure}[h]
    \centering
    \includegraphics[width=0.6\linewidth]{Images/plaatsvanarticulatie}
    \caption{Plaats van articulatie}
    \label{fig:articulatie}
\end{figure}

\subsection{Klinkers}
\begin{itemize}
    \item Geen blokkade
    \item Stemhebbend
    \item Gemiddelde duur: 70 ms
\end{itemize}

\begin{figure}[h]
    \centering
    \includegraphics[width=0.35\linewidth]{Images/klinkers.png}
    \caption{Klinkers}
    \label{fig:articulatie}
\end{figure}

\subsection{Co-articulatie}

\begin{figure}[h!]
    \centering
    \includegraphics[width=0.6\linewidth]{Images/coarticulatie.png}
    \caption{Co-articulatie}
    \label{fig:articulatie}
\end{figure}

\subsection{Socio-linguïstische variatie}
Medewerkers van diverse klassen warenhuizen wordt gevraagd op welke verdieping een product ligt, wetende dat het op de vierde verdieping ligt. Het bleek dat bij warenhuis Saks (chique) de [r] veel vaker wordt uitgesproken door de medewerkers dan in Klein (minder chique).
\begin{itemize}
    \item Dialect
    \item Geslacht, leeftijd, sociale status
    \begin{itemize}
        \item Bijv. mannen gebruiken vaker niet-standaard vormen
    \end{itemize}
    \item Identiteit
    \begin{itemize}
        \item Saks, Macy's, Klein: [r] in fourth floor
        \item Martha's Vineyard
    \end{itemize}
\end{itemize}
\begin{figure}[h]
    \centering
    \includegraphics[width=0.4\linewidth]{Images/sociolinguistischevariatie.png}
    \caption{Socio-linguïstische variatie}
    \label{fig:sociolinguistisch}
\end{figure}
\section{Audio}
\noindent
De audiogolf van een woord:\\
\begin{figure}[h]
    \centering
    \includegraphics[width=0.6\linewidth]{Images/audiogolf.png}
    \caption{Audiogolf}
    \label{fig:audiogolf}
\end{figure}
\newpage
\noindent Een sample (25 ms) van de audiogolf lijkt een patroon te bevatten:\\
\begin{figure}[h]
    \centering
    \includegraphics[width=0.6\linewidth]{Images/audiogolfsample.png}
    \caption{Audiogolf sample}
    \label{fig:sample}
\end{figure}

\noindent We voeren een Fourier transformatie uit op de sample:\\
\begin{figure}[h]
    \centering
    \includegraphics[width=0.6\linewidth]{Images/fouriertrans.png}
    \caption{Fourier transformatie}
    \label{fig:fouriertrans}
\end{figure}

\noindent Het audiospectrum van de Fourier transformatie op toon [iy]:\\
\begin{figure}[h]
    \centering
    \includegraphics[width=0.4\linewidth]{Images/shefourier.png}
    \caption{[iy] in 'She'}
    \label{fig:fouriertrans}
\end{figure}

\noindent Dit proces wordt elke 10 ms uitgevoerd. Dat wordt \textit{windowing} genoemd. Als alle audiospectra van de Fouriertransformatie achter elkaar geplakt worden ontstaat de volgende spectogram van het woord 'bericht':
\begin{figure}[h!]
    \centering
    \includegraphics[width=0.6\linewidth]{Images/spectogram.png}
    \caption{Spectogram}
    \label{fig:fouriertrans}
\end{figure}




}

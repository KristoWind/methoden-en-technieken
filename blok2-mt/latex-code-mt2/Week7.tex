{\large
\textbf{{\LARGE Week 7}}
\section{Deep Belief Networks}
Het idee van een DBN is dat de \textit{features} van een RBM gebruikt worden als invoer voor een volgende RBM, en zo een paar lagen diep.
\begin{itemize}
    \item In feite is dit een Boltzmann Machine met een zwakkere restrictie dan een RBM, er mogen nu wel correlaties zijn tussen \textit{hidden nodes}, maar niet in dezelfde laag. 
    \item De diepere lagen leren meer en meer abstracte features.
\end{itemize}
\noindent Voordeel is dat deze lagen 1 voor 1 getraind kunnen worden. Dit maakt het een van de historisch eerste \textit{deep learning} modellen die efficiënt getraind kon worden.\\

\begin{figure}[h]
    \centering
    \includegraphics[width=0.4\linewidth]{Images/DBNgraph.png}
    \caption{Deep Belief Network vs. Deep Boltzmann Machine}
    \label{fig:DBNvsDBM}
\end{figure}

\noindent We kunnen nieuwe afbeeldingen genereren door een oorspronkelijk beeld nu aan de \textit{visible} laag aan te bieden, de \textit{hidden} lagen een voor een te samplen, en dan terug te gaan en de \textit{visible} lagen te samplen.\\

\noindent We kunnen de gewichten en bias van een DBN gebruiken om een neuraal netwerk te initialiseren. Dit is een vorm van \textit{transfer learning}.

\begin{figure}[h]
    \centering
    \includegraphics[width=0.4\linewidth]{Images/DBNMNIST.png}
    \caption{Deep Belief Network on MNIST-dataset}
    \label{fig:DBN}
\end{figure}


% -semi sup (50000/100000, nepgelabeld) en sup (5000): voorspelmodel maken, dat vergeleken met classifier die getraind is op 5000 uit traindata

% vergelijking maken tussen sup en semi sup. classifier naar keuze (beslisboom/NN), 5000 trainset
% - Heeft dit model geholpen? Testen door met classifier op beide modellen te runnen en kijken wat beter werkt. 

% 5000, 10 per afbeelding genereren. 

% testen met bijvoorbeeld MNIST dataset. Mogen we zelf kiezen, ook hoeveel maakt niet uit in theorie.

% code van robert slides tot data genereren plaatjes. dan iets slims doen om niet 10 plaatjes maar heel veel meer plaatjes.


% iteratieleerdoel: met (niet teveel) woorden beschrijven wat volgende stap zou kunnen zijn.

}

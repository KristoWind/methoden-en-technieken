{\large
\textbf{{\LARGE Week 5}}
\section{Variational Auto-Encoders}
De auto-encoder probeert het kromme vlak te leren, waar de data op leeft. De PCA kan dat niet, die leeft op een lineair vlak. 

\begin{figure}[h]
    \centering
    \includegraphics[width=0.4\linewidth]{Images/AE_mt.png}
    \caption{$\mathbb{R}^2$ Auto-encoder}
    \label{fig:generatedfaces}
\end{figure}

\begin{itemize}
    \item Waarnemingen worden gegenereerd via een kansproces
    \item De latent space beschrijft niet de exacte data maar een kansverdeling van dit proces
    \item We nemen aan dat deze kansverdeling een normaal verdeling is
    \item De encoder leert gemiddelde ($\mu$) en variantie ($\sigma ^2$) van de verdeling te schatten
    \item Vervolgens wordt er een trekking gedaan uit deze verdeling
    \item De decoder probeert deze trekking terug te transformeren naar de oorspronkelijke data.
\end{itemize}
\noindent Omdat de decoder een variantie moet schatten is een aangepaste loss-functie nodig:
\[\textrm{loss}=\textrm{Reconstruction loss}+\textrm{\textit{KL}-loss}\]

\begin{figure}[h]
    \centering
    \includegraphics[width=0.6\linewidth]{Images/VAE_architectuur.png}
    \caption{Variational auto-encoder architectuur}
    \label{fig:generatedfaces}
\end{figure}

Variational auto-encoders geven niet alleen 2 coördinaten, maar ook 2 onzekerheden om het punt heen. Sampling zit alleen in het trainingsproces. Hij leert welke gemiddelde ($\mu$) en standaardafwijking ($\sigma$) goed zijn om de reconstructie-fout laag te hebben. Eigenlijk is het een soort verborgen data-augmentatie midden in je model. Nieuwe data genereren in je model om beter te trainen.


\begin{figure}[h]
    \centering
    \includegraphics[width=0.6\linewidth]{Images/VAE_mt.png}
    \caption{$\mathbb{R}^2$ Variational auto-encoder}
    \label{fig:generatedfaces}
\end{figure}


}
